\documentclass[presentation]{beamer}

\usecolortheme{Imperial}
 
\usepackage[utf8]{inputenc}
\usepackage[UKenglish]{babel}
\usepackage{booktabs}
\usepackage{caption}
\usepackage{subcaption}
\usepackage{graphicx}
\usepackage{amsmath}
\usepackage{amsfonts}
\usepackage{amssymb}
\usepackage{epstopdf}
\usepackage{csquotes}
\usepackage{tikz}
\usepackage{media9}

% complying UK date format, i.e. 1 January 2001
\usepackage{datetime}
\let\dateUKenglish\relax
\newdateformat{dateUKenglish}{\THEDAY~\monthname[\THEMONTH] \THEYEAR}

% Imperial College Logo, not to be changed!
\institute{\includegraphics[height=0.7cm]{Imperial_1_Pantone_solid.eps}}

\title{UROP 2023}

\author{Archibald Browne}

\begin{document}

\frame{\titlepage}

\begin{frame}{What did I do?}
    The project focused around formalising exercises from Professor M. Liebeck's 'A Concise Introduction to Mathematics' in Lean. Specifically, those from Chapter 10 'The Integers'.
    \pause
    \\ 
    \vspace{1em}
    The questions were mainly focused around the following areas:
    \begin{itemize}
        \item Greatest Common Divisor
        \item Lowest Common Multiple
        \item Bézout's Identity
        \item Prime Numbers
    \end{itemize}
    \pause
    I will go over some of the highlights/difficulties I experienced over the project, and what I learnt.
\end{frame}

\begin{frame}{Highlights}
    Here I will look at a couple of my favourite questions:
    \pause
    \begin{block}{Question 8}
        Let \(n \geq 2\) be an Integer. Prove that \(n\) is prime if and only if \(\forall a \in \mathbb{Z}, \  \gcd(a, n) = 1 \vee n \vert a \)
    \end{block}
    \pause
    \begin{itemize}
        \item Like many of the questions, this was much harder to teach to Lean than it was to solve. 
        \item The general idea is that prime numbers cannot have common factors with any numbers that aren't a multiple of that prime.
    \end{itemize}
\end{frame}

\begin{frame}{Highlights}
    Question Statement in Lean:
    \\
    \vspace{1em}
    \includegraphics[width=\textwidth]{UROP presentation/Question8.png}
    \pause
    \\
    \vspace{1em}
    Lean is fussy about the distinction between integers and naturals, so even though we have declared \(2 \leq n\) and \(n \in \mathbb{Z}\), we still need to tell Lean that \(n\) is a natural number in order to use theorems about natural numbers.
    \\
    \vspace{1em}
    \pause
    For the \(\implies\) direction, we condition on whether \( n \vert a\). If it does, we get the result immediately. If not, then \(gcd(a, n) = 1\) because \( n\) is prime.
\end{frame}

\begin{frame}[fragile]{Highlights}
    For the \(\impliedby\) direction, we use the theorem:
    \begin{verbatim}
    Nat.prime_def_lt'
    \end{verbatim}
    Which says:
    \[n \in \mathbb{N} \text{ is prime} \iff \forall d \in (1, n) \cap \mathbb{N}, d \nmid n\]
    \pause
    We then condition over our assumption specialized for some particular \(d\) value:
    \begin{itemize}
        \item If \(gcd(d, n) = 1\) then clearly \(d \nmid n\)
        \item If \(d \vert n\) then we have \(d \geq n \vee d = 1\), a contradiction
    \end{itemize}
    
\end{frame}


\begin{frame}{What did I Learn?}
    Specific
\end{frame}

\begin{frame}{What did I Learn?}
    General
\end{frame}


\begin{frame}{Conclusion}
    
\end{frame}
\end{document}